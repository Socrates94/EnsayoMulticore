%%%%% Fichero principal main.tex %%%%%%
\documentclass[twocolumn,twoside]{Jornadas}
\usepackage[utf8]{inputenc}
\usepackage[spanish]{babel}
\usepackage{listings}
\usepackage{algorithm}
\usepackage{algorithmicx}
\usepackage{algcompatible}
\usepackage{adjustbox}
\usepackage{graphicx}
\usepackage{color}
\usepackage{caption}
\captionsetup{font=footnotesize}
\usepackage[caption=false,font=footnotesize]{subfig}
\usepackage{placeins}
\usepackage{hyperref}
\usepackage{url}
\usepackage{amsmath} % <-- Añadido para que compile tu ecuación matemática

% Configuración de colores y bloques de código de tu plantilla original
\definecolor{gray97}{gray}{.97}
\definecolor{gray75}{gray}{.75}
\definecolor{gray45}{gray}{.45}

\lstset{
     inputencoding=utf8,
     extendedchars=true,
     backgroundcolor=\color{gray97},
     stringstyle=\ttfamily,
     showstringspaces = false,
     basicstyle=\scriptsize\ttfamily,
     commentstyle=\color{gray45},
     keywordstyle=\bfseries,
     linewidth=.98\columnwidth,
     xleftmargin=3mm,
     breaklines=true,
     numbers=left,
     numbersep=6pt,
     numberstyle=\tiny,
     numberfirstline = false,
     firstnumber=auto,
     escapeinside={(*@}{@*)},
     literate={á}{{\'a}}1 {é}{{\'e}}1 {í}{{\'i}}1 {ó}{{\'o}}1 {ú}{{\'u}}1 {ñ}{{\~n}}1
}

\def\BibTeX{{\rm B\kern-.05em{\sc i\kern-.025em b}\kern-.08em T\kern-.1667em\lower.7ex\hbox{E}\kern-.125emX}}

% AQUÍ CONFIGURAMOS TU CARPETA DE IMÁGENES
\graphicspath{{.}{./img/}} 
%%%%%%%%%%%%%%%%%%%%%%%%%%%%%%%%%%%%%%%%%%%%

\begin{document}

\title{Análisis de Rendimiento y Vectorización SIMD de un Kernel Numérico para Detección de Bordes basado en el Operador Sobel}

\author{%
     José Daniel Flores Morales%
     \thanks{Benemérita Universidad Autónoma de Puebla, Maestría en Sistemas Distribuidos, Facultad de Ciencias de la Computación. Correo: {\tt fm225470536@alm.buap.mx}}
}

\maketitle
\markboth{}{}
\pagestyle{empty} 
\thispagestyle{empty} 

\begin{abstract}
La detección de bordes es una etapa crítica en el preprocesamiento de imágenes médicas y navegación autónoma, exigiendo alta eficiencia computacional. Este artículo evalúa el impacto de la vectorización SIMD en el rendimiento de un kernel numérico basado en el operador Sobel. Se desarrollaron cuatro implementaciones en C++: escalar, vectorización automática, guiada (OpenMP) y explícita utilizando intrínsecas AVX2. Las pruebas se ejecutaron sobre un espacio de datos masivo equivalente a 800 iteraciones de una imagen de 256x256 píxeles. El perfilado en Intel Advisor confirmó la generación de instrucciones vectoriales de 256 bits. Sin embargo, el análisis del modelo Roofline revela que el speedup frente a la versión escalar es marginal. Se concluye que el cálculo posee baja intensidad aritmética, clasificando al kernel como Memory-Bound, limitando el rendimiento al ancho de banda de la memoria principal.
\end{abstract}

\begin{keywords}
Vectorización SIMD, AVX2, Auto-vectorización, OpenMP, Operador Sobel, Intel Advisor, Modelo Roofline, Memory-Bound.
\end{keywords}

% ARCHIVOS MODULARES (Se omiten el 01 y 02 porque ya están arriba)

%!TEX root = main.tex
\section{Marco Teórico}

\subsection{El Kernel Numérico y el Operador Sobel}
En el contexto de este trabajo, el \textbf{kernel numérico} se refiere a la región crítica de código responsable de aplicar el operador de detección de bordes. Específicamente, representa el ciclo computacionalmente intensivo que calcula la magnitud del gradiente para cada píxel mediante la siguiente ecuación:
\begin{equation}
    G[i] = \sqrt{G_{x}[i]^{2} + G_{y}[i]^{2}}
\end{equation}

Para obtener los gradientes direccionales ($G_x$ y $G_y$), se utilizó el \textbf{operador Sobel}, el cual emplea máscaras de convolución de $3\times3$. A diferencia del operador Canny, que introduce dependencias de datos que impiden la vectorización, Sobel permite el procesamiento independiente de cada píxel, haciéndolo ideal para la optimización SIMD.

\subsection{Vectorización SIMD y AVX2}
El paralelismo se evaluó utilizando las extensiones vectoriales avanzadas \textbf{AVX2} (Advanced Vector Extensions). Estos registros permiten empaquetar y procesar 8 datos de tipo \textit{float} (256 bits en total) de forma estrictamente simultánea en un solo ciclo de reloj. 

En la Figura \ref{fig:codigoAuto} se ilustra la implementación del kernel base sobre el cual se aplicaron las directivas de vectorización.

\begin{figure}[htb]
    \centering
    \begin{minipage}{\linewidth}
    {\footnotesize
    \begin{lstlisting}[mathescape=true]
Entrada: 
  Gx: Arreglo de gradientes horizontales
  Gy: Arreglo de gradientes verticales
  N:  Total de pixeles a procesar

Salida: 
  Mag: Arreglo con la magnitud resultante

Procedimiento Calculo_Magnitud(Gx, Gy, N, Mag):
    Para i desde 0 hasta N - 1 hacer:
        Mag[i] $\leftarrow \sqrt{Gx[i]^2 + Gy[i]^2}$
    Fin Para
Fin Procedimiento
    \end{lstlisting}
    }
    \end{minipage}
    \caption{Pseudocódigo del kernel numérico utilizado para el cálculo de la magnitud del gradiente.}
    \label{fig:codigoAuto}
\end{figure}

\subsection{Perfilado y Modelo Roofline}
Para el análisis de rendimiento se empleó la herramienta \textit{Intel Advisor}. Esta plataforma permite generar el modelo \textbf{Roofline}, una representación visual que correlaciona el rendimiento máximo teórico (medido en GFLOPS) con la intensidad aritmética del algoritmo (operaciones de punto flotante por byte transferido) y los límites de ancho de banda de la jerarquía de memoria del sistema.
%!TEX root = main.tex
\section{Introducción}

\subsection{Contexto del procesamiento de imágenes}
\PARstart{L}{a} detección de bordes es una operación fundamental en el procesamiento de imágenes digitales y la visión artificial. Esta técnica transforma una matriz de píxeles en información estructural vital, resaltando formas y contornos al identificar los cambios más bruscos de intensidad luminosa en la imagen original. Algoritmos clásicos, como el operador Sobel, son ampliamente utilizados por su simplicidad matemática y su eficacia para calcular la magnitud del gradiente bidimensional.

\subsection{Importancia del rendimiento en el procesamiento de imágenes}
En aplicaciones del mundo real, la velocidad computacional es un factor crítico. Tareas como el preprocesamiento de imágenes médicas o los sistemas de percepción en vehículos autónomos requieren respuestas en tiempo real. El procesamiento escalar tradicional suele ser insuficiente ante grandes volúmenes de datos, por lo que resulta indispensable aprovechar las capacidades de paralelismo a nivel de datos que ofrecen los procesadores modernos. 

En este artículo se analiza el impacto de aplicar diferentes técnicas de vectorización SIMD sobre un kernel de detección de bordes, con el objetivo de evaluar empíricamente los beneficios y las limitantes del hardware al procesar cargas de trabajo intensivas.
%!TEX root = main.tex
\section{Marco teórico}

\subsection{Vectorización SIMD}
La vectorización se basa en el modelo SIMD (Single Instruction, Multiple Data), que permite a un procesador ejecutar una misma operación matemática sobre un conjunto de datos simultáneamente. En este proyecto se utilizan las extensiones AVX2 (Advanced Vector Extensions), cuyos registros de 256 bits tienen la capacidad de empaquetar y procesar 8 datos de tipo \textit{float} de forma paralela.

\subsection{Herramienta de Perfilado Intel Advisor}
Para la evaluación de rendimiento se utiliza \textit{Intel Advisor}, una herramienta avanzada de diseño y análisis de vectorización. Su característica más destacada es el modelo \textbf{Roofline}, el cual correlaciona de manera visual el rendimiento máximo (medido en GFLOPS) con la intensidad aritmética del algoritmo y los límites físicos del ancho de banda de la memoria del sistema.

\subsection{El Operador Sobel y Máscaras de Convolución}
El operador Sobel es un filtro discreto de diferenciación computacional que calcula una aproximación del gradiente de la función de intensidad de la imagen. Se basa en la convolución de la matriz de píxeles original con dos filtros (máscaras) de $3 \times 3$, uno diseñado para detectar cambios de luminosidad horizontales ($G_x$) y otro para cambios verticales ($G_y$). 

Dichas matrices de convolución están definidas matemáticamente de la siguiente manera:

\begin{equation}
G_x = \begin{bmatrix} -1 & 0 & +1 \\ -2 & 0 & +2 \\ -1 & 0 & +1 \end{bmatrix} * A 
\quad \text{y} \quad 
G_y = \begin{bmatrix} +1 & +2 & +1 \\ 0 & 0 & 0 \\ -1 & -2 & -1 \end{bmatrix} * A
\end{equation}

Donde $A$ representa la región de $3 \times 3$ píxeles de la imagen de origen. Al aplicar estas máscaras, el algoritmo extrae la magnitud del gradiente, la cual representa la "fuerza" o nitidez del borde en cada píxel. La combinación de ambos gradientes ortogonales mediante la norma euclidiana constituye el núcleo iterativo (\textit{kernel}) que fue perfilado a nivel de hardware en esta investigación.
%!TEX root = main.tex
\section{Metodología}

\subsection{Descripción del kernel}
El kernel numérico objetivo de este estudio calcula la magnitud del gradiente de una imagen utilizando el operador Sobel. Se perfiló exclusivamente el ciclo intensivo definido por la siguiente ecuación matemática:
\begin{equation}
    G[i] = \sqrt{G_{x}[i]^{2} + G_{y}[i]^{2}}
\end{equation}

\subsection{Plataforma experimental}
Para contextualizar correctamente los resultados arrojados por el perfilador y entender los topes del modelo Roofline, es imperativo definir la arquitectura física anfitriona (\textit{host}). Las pruebas de estrés y validación se ejecutaron bajo el siguiente entorno de hardware y software, procesando una imagen PGM sintética de $256 \times 256$ píxeles, replicada en memoria 800 veces para generar una carga de trabajo robusta:

\begin{itemize}
    \item \textbf{Procesador (CPU):} AMD Ryzen 9 5950X (32 hilos lógicos a 4.95 GHz). Arquitectura de 64 bits de alto rendimiento con soporte nativo para el conjunto de instrucciones de extensiones vectoriales avanzadas (AVX y AVX2).
    \item \textbf{Memoria Principal (RAM):} 128 GB de capacidad total de almacenamiento volátil, implementando tecnología DDR4 a 3200 MHz. Como se evidenció en los perfiles, el sistema opera con un ancho de banda pico real medido de entre $29.06$ y $31.48$ GB/s.
    \item \textbf{Jerarquía de Memoria Caché:} Sistema de caché escalonado (L1, L2 y L3) integrado en el silicio para la mitigación de latencias, aunque insuficiente para retener la carga completa de la imagen.
    \item \textbf{Sistema Operativo:} Entorno basado en núcleo UNIX/Linux (Ubuntu 24.04.4 LTS x86\_64), garantizando una mínima sobrecarga de procesos en segundo plano durante el perfilado.
    \item \textbf{Suite de Desarrollo:} \textit{Intel oneAPI Base Toolkit} versión 2023, utilizando el compilador C++ (\texttt{icpx}) y el analizador de vectorización (\textit{Intel Advisor}).
\end{itemize}

\subsection{Procedimiento de Compilación y Configuración de Perfilado}
Para garantizar la validez matemática y la precisión de las métricas recolectadas por \textit{Intel Advisor}, el proceso de compilación requirió una configuración minuciosa de banderas (\textit{flags}). El objetivo principal fue habilitar la recolección de métricas de hardware sin que las optimizaciones agresivas de la herramienta ofuscaran el análisis del código.

A continuación, se detallan los comandos de compilación ejecutados mediante la terminal:

\begin{itemize}
    \item \textbf{Versión Escalar:} Se forzó la omisión de paralelismo y se deshabilitaron los reportes vectoriales para mantener un binario estrictamente secuencial.
    \begin{lstlisting}[language=bash]
icpx -g -O3 -march=native -fno-inline -o scalar KernelScalar.cpp
    \end{lstlisting}

    \item \textbf{Versión Automática:} Se habilitaron los reportes de vectorización para auditar las decisiones autónomas del compilador.
    \begin{lstlisting}[language=bash]
icpx -g -O3 -march=native -fno-inline -qopt-report=max -qopt-report-phase=vec -o auto KernelAuto.cpp
    \end{lstlisting}

    \item \textbf{Versión Guiada:} Se incluyó la directiva necesaria para compilar el estándar de multiprocesamiento simétrico.
    \begin{lstlisting}[language=bash]
icpx -g -O3 -march=native -qopenmp -fno-inline -qopt-report=max -qopt-report-phase=vec -o guiada KernelGuiada.cpp
    \end{lstlisting}

    \item \textbf{Versión Explícita (AVX2):} Se mantuvo la configuración de auditoría vectorial para validar la emisión de instrucciones de 256 bits.
    \begin{lstlisting}[language=bash]
icpx -g -O3 -march=native -fno-inline -qopt-report=max -qopt-report-phase=vec -o explicit KernelExplicit.cpp
    \end{lstlisting}
\end{itemize}

\subsubsection{Justificación de los Parámetros del Compilador}
El uso de la sintaxis anterior responde a exigencias estrictas del perfilado de hardware:
\begin{itemize}
    \item \texttt{-g}: Genera los símbolos de depuración, imperativo para que Intel Advisor logre mapear los cuellos de botella detectados directamente hacia las líneas de código fuente en C++.
    \item \texttt{-march=native}: Instruye al compilador a exprimir el conjunto de instrucciones máximo soportado por la arquitectura del procesador anfitrión, habilitando la emisión de instrucciones AVX2.
    \item \texttt{-fno-inline}: Evita que el compilador expanda las funciones directamente en la rutina principal (\textit{main}), aislando el kernel del operador Sobel para medir exclusivamente su tiempo de ejecución sin ruido de E/S.
    \item \texttt{-qopt-report}: Genera registros textuales de máxima verbosidad sobre la fase de vectorización, crucial para comprender por qué el compilador aprueba o rechaza paralelizar un ciclo.
\end{itemize}

\subsubsection{Flujo de Trabajo en Intel Advisor}
Una vez generados los ejecutables, el análisis de rendimiento consistió en una recolección bifásica utilizando la interfaz de \textit{Intel Advisor}:
\begin{enumerate}
    \item \textbf{Survey Analysis (Análisis de Inspección):} Se ejecutó inicialmente para medir los tiempos de ejecución base, detectar los bucles más pesados y generar el \textit{Instruction Mix}.
    \item \textbf{Trip Counts and FLOP Analysis (Análisis de Iteraciones y Operaciones):} Se instrumentó el binario para contar las operaciones matemáticas exactas de punto flotante y medir el tráfico del bus de datos. Esta fase inyecta los datos en el modelo \textbf{Roofline}.
\end{enumerate}
\FloatBarrier

\subsection{Versiones implementadas}
Para evaluar el impacto de las diferentes técnicas de paralelismo a nivel de datos, se desarrollaron cuatro variantes del kernel en C++.

% bloque scalar
\subsubsection{Versión Escalar (Línea Base)}
La versión escalar representa el algoritmo tradicional secuencial. Su propósito es establecer el punto de referencia de rendimiento ($1.00\times$) para evaluar las posteriores optimizaciones SIMD. En esta implementación, se omitió cualquier directiva de vectorización y se forzó al compilador a generar instrucciones escalares, procesando la imagen de entrada píxel por píxel.

\begin{figure}[htb]
    \centering
    \begin{minipage}{\linewidth}
    {\footnotesize
    \begin{lstlisting}[mathescape=true]
Entrada: 
  Gx: Arreglo de gradientes horizontales
  Gy: Arreglo de gradientes verticales
  N:  Cantidad total de pixeles a procesar

Salida: 
  Mag: Arreglo con la magnitud resultante

Procedimiento Operador_Sobel_Escalar(Gx, Gy, N, Mag):
    // Procesamiento secuencial (un pixel por iteracion)
    Para i desde 0 hasta N - 1 hacer:
        Mag[i] $\leftarrow \sqrt{Gx[i]^2 + Gy[i]^2}$
    Fin Para
Fin Procedimiento
    \end{lstlisting}
    }
    \end{minipage}
    \caption{Pseudocódigo del kernel numérico para la versión escalar.}
    \label{fig:codigo_escalar}
\end{figure}

Al ejecutar este kernel sobre el conjunto de datos de prueba, se obtuvo un tiempo de ejecución base en la terminal, validando simultáneamente la correcta generación de la imagen de salida.

\begin{figure}[htb]
    \centering
    \includegraphics[width=\linewidth]{scalar01.png}
    \caption{Salida en consola de la ejecución escalar, registrando un tiempo de 0.0273 segundos.}
    \label{fig:terminal_escalar}
\end{figure}

Para comprender el comportamiento a nivel de hardware, se perfiló la ejecución utilizando \textit{Intel Advisor}. El modelo Roofline resultante (Figura \ref{fig:roofline_escalar}) revela que el punto de ejecución (marcado en rojo) se encuentra muy por debajo del rendimiento matemático máximo del procesador (\textit{Scalar Add Peak}). Además, se ubica en la región izquierda de la gráfica, chocando contra el límite del ancho de banda de la memoria DRAM (Memory-Bound).

\begin{figure}[htb]
    \centering
    \includegraphics[width=\linewidth]{scalar02.png}
    \caption{Modelo Roofline de la versión escalar. El rendimiento choca contra el límite de ancho de banda de la memoria.}
    \label{fig:roofline_escalar}
\end{figure}

El análisis detallado de las métricas del programa confirma la ineficiencia del procesamiento secuencial para esta carga de trabajo. Como se documenta en la Figura \ref{fig:metricas_escalar}, el 94.9\% del tiempo de ejecución se gasta en código escalar, produciendo un rendimiento sumamente bajo de apenas 0.53 GFLOPS.

\begin{figure}[htb]
    \centering
    \includegraphics[width=\linewidth]{scalar03.png}
    \caption{Métricas generales del programa perfiladas en Intel Advisor.}
    \label{fig:metricas_escalar}
\end{figure}

Finalmente, la inspección de la mezcla de instrucciones dinámicas (Figura \ref{fig:instrucciones_escalar}) corrobora empíricamente que el compilador respetó las restricciones de no-vectorización. La inmensa mayoría de las operaciones pertenecen a instrucciones de memoria (Memory) y procesamiento unario (Scalar/x86), sin aprovechamiento de los registros AVX2.

\begin{figure}[htb]
    \centering
    \includegraphics[width=\linewidth]{scalar04.png}
    \caption{Resumen de la mezcla de instrucciones (\textit{Instruction Mix}) ejecutadas por el CPU.}
    \label{fig:instrucciones_escalar}
\end{figure}
\FloatBarrier

% bloque automatica
\subsubsection{Versión Automática}

A nivel algorítmico, esta implementación es idéntica a la versión escalar. La optimización no se realiza inyectando código de bajo nivel, sino delegando la responsabilidad de la vectorización al compilador \texttt{icpx}. Para habilitar esto, se utilizó el nivel de optimización \texttt{-O3} y se agregaron calificadores de tipo \texttt{\_\_restrict} a los arreglos de entrada y salida, garantizando al compilador la ausencia de solapamiento de memoria (\textit{aliasing}).

\begin{figure}[htb]
    \centering
    \begin{minipage}{\linewidth}
    {\footnotesize
    \begin{lstlisting}[mathescape=true]
Entrada: 
  Gx, Gy: Arreglos restrictos de gradientes (sin aliasing)
  N: Cantidad total de pixeles a procesar

Salida: 
  Mag: Arreglo restricto con la magnitud resultante

Procedimiento Operador_Sobel_Auto(Gx, Gy, N, Mag):
    // El compilador decide como agrupar las iteraciones
    Para i desde 0 hasta N - 1 hacer:
        Mag[i] $\leftarrow \sqrt{Gx[i]^2 + Gy[i]^2}$
    Fin Para
Fin Procedimiento
    \end{lstlisting}
    }
    \end{minipage}
    \caption{Pseudocódigo del kernel numérico para la versión automática. La lógica es idéntica, pero el compilador asume la tarea de paralelizar.}
    \label{fig:codigo_auto}
\end{figure}

Al ejecutar el binario, el tiempo reportado en consola (0.0277 segundos) fue prácticamente indistinguible de la línea base.

\begin{figure}[htb]
    \centering
    \includegraphics[width=\linewidth]{auto01.png}
    \caption{Salida en consola de la ejecución automática, registrando un tiempo de 0.0277 segundos.}
    \label{fig:terminal_auto}
\end{figure}

Para comprender la falta de \textit{speedup}, se analizó el comportamiento del binario en Intel Advisor. Las métricas generales (Figura \ref{fig:metricas_auto}) revelan un comportamiento sumamente conservador por parte del compilador: el 92.5\% del tiempo de ejecución se consumió procesando código puramente escalar, logrando vectorizar únicamente una fracción minúscula del ciclo de trabajo (7.5\%).

\begin{figure}[htb]
    \centering
    \includegraphics[width=\linewidth]{auto02.png}
    \caption{Métricas del programa donde se evidencia que el compilador no logró vectorizar eficazmente el ciclo principal.}
    \label{fig:metricas_auto}
\end{figure}

Esta ineficiencia se refleja claramente en el modelo Roofline (Figura \ref{fig:roofline_auto}). El punto de ejecución de esta versión ni siquiera logró superar el rendimiento del pico escalar (\textit{Scalar Add Peak} de 9.49 GFLOPS), estancándose en un rendimiento computacional muy pobre y chocando prematuramente contra el límite del ancho de banda de la memoria DRAM.

\begin{figure}[htb]
    \centering
    \includegraphics[width=\linewidth]{auto04.png}
    \caption{Modelo Roofline de la versión automática. El rendimiento se mantiene estancado por debajo del techo escalar.}
    \label{fig:roofline_auto}
\end{figure}

Finalmente, el análisis de código (Figura \ref{fig:instrucciones_auto}) demuestra que las instrucciones SIMD pesadas (como AVX2) representan apenas un 1\% del total de la mezcla dinámica. Esto comprueba que, ante ciclos con operaciones matemáticas ligeras y accesos a memoria pesados como el del operador Sobel, el auto-vectorizador del compilador falla en emitir código óptimo, haciendo necesaria la intervención manual del programador.

\begin{figure}[htb]
    \centering
    \includegraphics[width=\linewidth]{auto05.png}
    \caption{Análisis de código e \textit{Instruction Mix}, confirmando la casi nula presencia de instrucciones vectoriales.}
    \label{fig:instrucciones_auto}
\end{figure}
\FloatBarrier

% bloque guiada
\subsubsection{Versión Guiada (OpenMP)}
La versión guiada tiene como objetivo superar el conservadurismo del auto-vectorizador del compilador. Para lograrlo, se intervino explícitamente el código fuente utilizando el estándar OpenMP. Al inyectar la directiva \texttt{\#pragma omp simd} justo antes del ciclo principal, se le ordena al compilador que asuma la total independencia de las iteraciones y genere instrucciones vectoriales de manera obligatoria, ignorando las dependencias de datos que él creyera detectar.

\begin{figure}[htb]
    \centering
    \begin{minipage}{\linewidth}
    {\footnotesize
    \begin{lstlisting}[mathescape=true]
Entrada: 
  Gx, Gy: Arreglos restrictos de gradientes
  N: Cantidad total de pixeles a procesar

Salida: 
  Mag: Arreglo restricto con la magnitud resultante

Procedimiento Operador_Sobel_Guiado(Gx, Gy, N, Mag):
    // Directiva OpenMP para forzar la vectorizacion SIMD
    #pragma omp simd
    Para i desde 0 hasta N - 1 hacer:
        Mag[i] $\leftarrow \sqrt{Gx[i]^2 + Gy[i]^2}$
    Fin Para
Fin Procedimiento
    \end{lstlisting}
    }
    \end{minipage}
    \caption{Pseudocódigo del kernel numérico para la versión guiada, destacando el uso de la directiva de OpenMP.}
    \label{fig:codigo_guiado}
\end{figure}

Al compilar (asegurando el soporte para OpenMP) y ejecutar el binario, el tiempo en consola fue de 0.0284 segundos. Lejos de presentar una mejora, el tiempo se mantuvo en el mismo rango de tolerancia que las versiones anteriores, reafirmando que el cuello de botella no reside en la emisión de instrucciones.

\begin{figure}[htb]
    \centering
    \includegraphics[width=\linewidth]{guiada01.png}
    \caption{Salida en consola de la ejecución guiada, registrando un tiempo de 0.0284 segundos.}
    \label{fig:terminal_guiada}
\end{figure}

El reporte de métricas de Intel Advisor (Figura \ref{fig:metricas_guiada}) expone que, a pesar de la directiva explícita, el tiempo invertido en el ciclo vectorizado sigue siendo marginal (7.5\%). El procesador continúa destinando la abrumadora mayoría de sus ciclos de reloj (92.5\%) a gestionar código escalar, evidenciando que el pragma por sí solo no resuelve la inanición de datos de las unidades aritméticas.

\begin{figure}[htb]
    \centering
    \includegraphics[width=\linewidth]{guiada02.png}
    \caption{Métricas del programa donde se observa que el pragma de OpenMP no logró invertir la proporción de tiempo escalar frente al vectorizado.}
    \label{fig:metricas_guiada}
\end{figure}

Esta limitante física queda plasmada empíricamente en el modelo Roofline (Figura \ref{fig:roofline_guiada}). El punto de rendimiento de la versión guiada se sitúa sobre la línea diagonal de la memoria DRAM, demostrando que el algoritmo choca contra el muro del ancho de banda antes de poder escalar hacia el techo del rendimiento matemático (\textit{Scalar Add Peak}).

\begin{figure}[htb]
    \centering
    \includegraphics[width=\linewidth]{guiada04.png}
    \caption{Modelo Roofline de la versión guiada, confirmando su naturaleza estrictamente \textit{Memory-Bound}.}
    \label{fig:roofline_guiada}
\end{figure}

Finalmente, la analítica de código (Figura \ref{fig:instrucciones_guiada}) ratifica este diagnóstico. La penalización por acceder a la memoria principal es tan severa que las instrucciones de carga/almacenamiento y las operaciones escalares dominan el tiempo de ejecución. La arquitectura se ve imposibilitada de alimentar registros de 256 bits a la velocidad que requeriría un \textit{speedup} efectivo.

\begin{figure}[htb]
    \centering
    \includegraphics[width=\linewidth]{guiada05.png}
    \caption{Análisis de código de la versión guiada, dominado por el castigo de latencia en memoria.}
    \label{fig:instrucciones_guiada}
\end{figure}
\FloatBarrier

% bloque explicita
\subsubsection{Versión Explícita (AVX2)}

Esta última iteración representa el nivel más profundo de optimización manual. Para evadir por completo las decisiones conservadoras del compilador, se reprogramó el kernel utilizando funciones intrínsecas de Intel (\texttt{<immintrin.h>}). El ciclo \texttt{for} principal fue rediseñado con un incremento de 8 en 8, procesando bloques de 8 píxeles simultáneamente utilizando registros de 256 bits, mediante operaciones de carga (\texttt{\_mm256\_loadu\_ps}), multiplicación (\texttt{\_mm256\_mul\_ps}), suma (\texttt{\_mm256\_add\_ps}), raíz cuadrada (\texttt{\_mm256\_sqrt\_ps}) y almacenamiento (\texttt{\_mm256\_storeu\_ps}).

La lógica del algoritmo modificado se describe en el siguiente pseudocódigo, donde se evidencia el empaquetado vectorial y el manejo de residuos escalares para tamaños de imagen que no sean múltiplos exactos de 8:

\begin{figure}[htb]
    \centering
    \begin{minipage}{\linewidth}
    {\footnotesize
    \begin{lstlisting}[mathescape=true]
Entrada: 
  Gx, Gy: Arreglos de gradientes
  N: Cantidad total de pixeles a procesar

Salida: 
  Mag: Arreglo con la magnitud resultante

Procedimiento Operador_Sobel_Explicito(Gx, Gy, N, Mag):
    i $\leftarrow$ 0
    // Procesamiento vectorial SIMD (8 datos por ciclo)
    Para i desde 0 hasta N - 8 con paso 8 hacer:
        Vector_X $\leftarrow$ Cargar_8_Datos(Gx[i ... i+7])
        Vector_Y $\leftarrow$ Cargar_8_Datos(Gy[i ... i+7])

        Vector_X2 $\leftarrow$ Multiplicar_SIMD(Vector_X, Vector_X)
        Vector_Y2 $\leftarrow$ Multiplicar_SIMD(Vector_Y, Vector_Y)
        Suma_Vec $\leftarrow$ Sumar_SIMD(Vector_X2, Vector_Y2)

        Res_Vec $\leftarrow \sqrt{Suma\_Vec}$  // Raiz cuadrada vectorial
        Almacenar_8_Datos(Mag[i ... i+7], Res_Vec)
    Fin Para

    // Procesamiento escalar para pixeles residuales
    Mientras i < N hacer:
        Mag[i] $\leftarrow \sqrt{Gx[i]^2 + Gy[i]^2}$
        i $\leftarrow$ i + 1
    Fin Mientras
Fin Procedimiento
    \end{lstlisting}
    }
    \end{minipage}
    \caption{Pseudocódigo del kernel explícito, ilustrando el procesamiento por bloques mediante registros AVX2.}
    \label{fig:codigo_explicito}
\end{figure}

Al ejecutar el código, el tiempo reportado en consola (0.0274 segundos) fue prácticamente idéntico al de la línea base escalar.

\begin{figure}[htb]
    \centering
    \includegraphics[width=\linewidth]{explicit01.png}
    \caption{Salida en consola de la ejecución con intrínsecas AVX2, registrando 0.0274 segundos.}
    \label{fig:terminal_explicita}
\end{figure}

Para revelar por qué la inyección manual de instrucciones SIMD no redujo el tiempo de ejecución global, se consultaron las métricas de Intel Advisor (Figura \ref{fig:metricas_explicita}). Aunque el ciclo principal vectorizado se ejecuta con una rapidez extrema, el tiempo total del programa sigue dominado por las tareas no vectorizables de fondo y la gestión de memoria.

\begin{figure}[htb]
    \centering
    \includegraphics[width=\linewidth]{explicit02.png}
    \caption{Métricas generales del programa. El impacto del ciclo vectorizado se diluye en el tiempo global de la aplicación.}
    \label{fig:metricas_explicita}
\end{figure}

El diagnóstico definitivo se consolida al analizar el modelo Roofline (Figura \ref{fig:roofline_explicita_final}). El punto de ejecución alcanzó un rendimiento matemático formidable de \textbf{68.76 GFLOPS}, superando por casi un factor de $7\times$ el límite máximo escalar (\textit{Scalar Add Peak} de 9.49 GFLOPS). Esto prueba de manera irrefutable que el hardware SIMD fue estimulado correctamente por el código explícito.

No obstante, la herramienta reporta una intensidad aritmética crítica de tan solo \textbf{0.069 FLOP/Byte}. El punto de ejecución choca de frente contra la rampa diagonal etiquetada como \textit{DRAM Bandwidth (29.06 GB/sec)}.

\begin{figure}[htb]
    \centering
    \includegraphics[width=\linewidth]{explicit05.png}
    \caption{Modelo Roofline de la versión explícita. El kernel alcanza casi 70 GFLOPS pero satura instantáneamente el ancho de banda de la memoria DRAM.}
    \label{fig:roofline_explicita_final}
\end{figure}

Como conclusión de este perfilado, se demuestra empíricamente que el operador Sobel es un algoritmo estrictamente dominado por el ancho de banda de la memoria principal (\textit{Memory-Bound}). La velocidad de cómputo del procesador y las capacidades de las extensiones AVX2 superan holgadamente la capacidad física del bus de memoria para suministrar datos, impidiendo cualquier \textit{speedup} real a nivel de aplicación.

\FloatBarrier
%!TEX root = main.tex
\section{Resultados}

\subsection{Tiempos de ejecución}
Se midió el tiempo exclusivo del kernel numérico (ignorando I/O de archivos). Como se observa en la Tabla \ref{tab:tiempos}, los tiempos son casi idénticos en todas las versiones.

\begin{table}[htb]
\caption{Comparativa de Tiempos de Ejecución reales del Kernel.}
\label{tab:tiempos}
\begin{center}
\begin{tabular}{|l||c|c|}\hline
Versión & Tiempo (s) & Speedup \\\hline\hline
Escalar & 0.0273 & 1.00x \\\hline
Automática & 0.0277 & 0.98x \\\hline
Guiada & 0.0284 & 0.97x \\\hline
Explícita (AVX2) & 0.0274 & 0.99x \\\hline
\end{tabular}
\end{center}
\end{table}

\subsection{GFLOPS}
El reporte de Intel Advisor confirmó la correcta generación de instrucciones SIMD para las versiones optimizadas. Sin embargo, la métrica de GFLOPS (Giga Flops Per Second) se mantuvo notablemente baja, lejos del pico máximo teórico de la arquitectura AMD utilizada.

\subsection{Gráficas Roofline de Advisor}
La siguiente figura ilustra el análisis visual del rendimiento del kernel bajo el modelo Roofline.

\begin{figure}[htb]
    \centering
    %\includegraphics[width=\linewidth]{TU_IMAGEN_ROOFLINE.png}
    \caption{Modelo Roofline generado por Intel Advisor para la versión vectorizada.}
    \label{fig:roofline}
\end{figure}
%!TEX root = main.tex
\section{Discusión}

\subsection{Comparación entre vectorizaciones}
A nivel de precisión algorítmica, todas las implementaciones procesaron correctamente la matriz de la imagen PGM, generando los gradientes y contornos esperados sin pérdida de fidelidad visual ni matemática. Sin embargo, el rendimiento dinámico expuso contrastes severos entre la teoría y la práctica del paralelismo a nivel de datos. 

Mientras que el compilador, en sus versiones Automática y Guiada (OpenMP), optó por un enfoque conservador al detectar latencias de memoria (manteniendo el rendimiento estancado por debajo del pico escalar de 9.49 GFLOPS), la versión Explícita (AVX2) forzó exitosamente el uso de las unidades vectoriales. No obstante, a pesar de que esta última versión maximizó el empaquetado en registros de 256 bits y disparó el rendimiento de las ALUs, el \textit{speedup} global empírico frente a la versión escalar fue estadísticamente nulo (oscilando entre $0.98\times$ y $1.00\times$).

\subsection{Interpretación de los resultados y Diagnóstico}
La discrepancia entre el alto procesamiento matemático alcanzado por las intrínsecas y la falta de aceleración en tiempo real se explica íntegramente a través del modelo Roofline. El cálculo del operador Sobel presenta una intensidad aritmética críticamente baja, perfilada en 0.069 FLOP/Byte. Por cada cálculo matemático simple ejecutado, el procesador está obligado a transportar grandes volúmenes de datos desde y hacia la memoria principal (lectura de operandos $G_x$ y $G_y$, y escritura del resultado de la magnitud).

Como consecuencia, los puntos de ejecución en las gráficas nunca logran escalar hacia los "techos" horizontales de capacidad computacional máxima. En su lugar, colisionan prematuramente contra las líneas diagonales que dictan el límite físico del hardware: el ancho de banda de la memoria DRAM (saturado a $\sim$31 GB/s). 

Este fenómeno provoca inanición de datos (\textit{data starvation}) en los registros SIMD; el procesador es capaz de calcular 8 píxeles simultáneamente en una fracción de ciclo de reloj, pero pasa la inmensa mayoría del tiempo inactivo, esperando a que el bus de memoria entregue el siguiente bloque de datos. Por lo tanto, se concluye categóricamente que el kernel evaluado es un algoritmo estrictamente \textit{Memory-Bound}. Para obtener mejoras tangibles de rendimiento en trabajos futuros, la estrategia deberá girar hacia técnicas que maximicen la localidad de datos (como \textit{loop tiling}) para explotar la memoria caché L1/L2, reduciendo el costoso tráfico hacia la RAM.
%!TEX root = main.tex
\section{Conclusiones}
Este análisis experimental demuestra que la optimización de código mediante vectorización explícita AVX2 es una herramienta poderosa, pero su efectividad está condicionada por la arquitectura de memoria. En kernels iterativos de baja intensidad aritmética como el operador Sobel, el cuello de botella físico impide alcanzar el speedup teórico del paralelismo SIMD. Para obtener mejoras significativas de rendimiento en este tipo de procesamiento de imágenes, las futuras investigaciones deberán priorizar técnicas que incrementen la localidad de datos y optimicen el uso de la memoria caché.

Este análisis experimental demuestra que la optimización de código mediante vectorización explícita AVX2 es una herramienta poderosa, pero su efectividad está estrictamente condicionada por la arquitectura de memoria. 

En kernels iterativos de baja intensidad aritmética como el operador Sobel, el cuello de botella físico impide alcanzar el \textit{speedup} teórico del paralelismo SIMD. Se comprobó que la limitante no es la capacidad total de la memoria principal (RAM), sino su ancho de banda; el bus de la tecnología DDR4 del equipo de pruebas alcanzó su límite físico de transferencia ($\sim$29 GB/s), saturando la alimentación de datos hacia el procesador. 

Para obtener mejoras significativas de rendimiento en este tipo de procesamiento de imágenes, no basta con acelerar las unidades aritmético-lógicas (ALUs), sino que las futuras investigaciones deberán priorizar técnicas que incrementen la localidad de datos (como \textit{tiling} o \textit{loop blocking}) para optimizar el uso de la memoria caché L1/L2 y evitar el costoso viaje a la RAM.

% REFERENCIAS APA 7 (Deben ir ANTES del end)
\nocite{*}
\bibliographystyle{apalike}
\bibliography{biblio}

\end{document}