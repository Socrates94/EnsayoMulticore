%!TEX root = main.tex

\begin{abstract}
La detección de bordes es una etapa crítica en el preprocesamiento de imágenes médicas y navegación autónoma, exigiendo alta eficiencia computacional. Este artículo evalúa el impacto de la vectorización SIMD en el rendimiento de un kernel numérico basado en el operador Sobel. Se desarrollaron cuatro implementaciones en C++: escalar, vectorización automática, guiada (OpenMP) y explícita utilizando intrínsecas AVX2. Las pruebas se ejecutaron sobre un espacio de datos masivo equivalente a 800 iteraciones de una imagen de $256 \times 256$ píxeles. El perfilado en Intel Advisor confirmó la correcta emisión de instrucciones vectoriales de 256 bits; sin embargo, el análisis del modelo Roofline revela que el speedup frente a la versión escalar es estadísticamente nulo. Se concluye empíricamente que el cálculo posee una intensidad aritmética críticamente baja, clasificando al kernel como estrictamente \textit{Memory-Bound} y limitando su rendimiento al ancho de banda físico de la memoria principal.
\end{abstract}

\begin{keywords}
Vectorización SIMD, AVX2, Auto-vectorización, OpenMP, Operador Sobel, Intel Advisor, Modelo Roofline, Memory-Bound.
\end{keywords}